
\documentclass[num-refs]{wiley-networks}
\usepackage{enumerate}
\usepackage{interval}
\usepackage{float}
\intervalconfig{left open fence= (,right open fence=)}
\usepackage[ruled,vlined]{algorithm2e}
\usepackage{standalone}
\usepackage{tikz}
\usetikzlibrary{arrows,shapes,positioning}
\usepackage{booktabs} % For \toprule, \midrule and \bottomrule
\usepackage{pgfplotstable} % Generates table from .csv
\usepackage[normalem]{ulem}
\newcommand{\stkout}[1]{\ifmmode\text{\sout{\ensuremath{#1}}}\else\sout{#1}\fi}
\usepackage{soul}
\pgfplotsset{compat=newest}

% Update article type if known
\papertype{Original Article}
\paperfield{Networks}

% Data of csv files
% \pgfplotstableread[col sep=comma]{data/new_results_pm.csv}\summarypm{}
% \pgfplotstableread[col sep=comma]{data/results_tue2019_latex_overall.csv}\overallpm{}
\pgfplotstableread[col sep=comma]{/home/daniel/CG_results_20191004/CG_summary_20191004.csv}\summary{}
\pgfplotstableread[col sep=comma]{/home/daniel/CG_results_20191004/CG_allinstances_20191004.csv}\allinstances{}

% \theoremstyle{TH}
\newtheorem{property}{Property}
\makeatletter
\newcommand\footnoteref[1]{\protected@xdef{}\@thefnmark{\ref{#1}}\@footnotemark}
\makeatother

\title{A new formulation for parallel machine scheduling problems using ZDDs}

\author[1\authfn{1}]{Daniel Kowalczyk}
\author[2\authfn{2}]{Roel Leus}

% Include full affiliation details for all authors
\affil[1]{Combinatorial Optimization Group, Eindhoven University of Technology, Eindhoven, 5600 MB, The Netherlands}
\affil[2]{ORSTAT, KU Leuven, Leuven, Leuven, Belgium}

\corraddress{Roel Leus, ORSTAT, KU Leuven, Leuven, Leuven, Belgium}
\corremail{roel.leus@kuleuven.be}

\fundinginfo{Funder One, Funder One Department, Grant/Award Numbers: 123456 and 123458; Funder Two, Funder Two Department, Grant/Award Number: 123459}

% Include the name of the author that should appear in the running header
\runningauthor{Daniel Kowalczyk and Roel Leus}
\begin{document}

\begin{table}[H]
	\centering
	\begin{threeparttable}
		\caption{Summary of results for algorithms \textbf{TIF}, \textbf{ATIF} and \textbf{BDDF}: Size of graph}\label{tbl:sizetw}
		\pgfplotstabletypeset[
			columns={n,m,first_size_graph_mean_TI,first_size_graph_amax_TI,first_size_graph_mean_ATI,first_size_graph_amax_ATI,first_size_graph_mean_AFBC,first_size_graph_amax_AFBC,reduction_mean_AFBC},
			every head row/.style={
					before row={%
							\toprule
							\multicolumn{2}{c}{}&  \multicolumn{2}{c}{\textbf{TIF}}& \multicolumn{2}{c}{\textbf{ATIF}} & \multicolumn{3}{c}{\textbf{BDDF}}\\
							\cmidrule(lr){3-4}\cmidrule(lr){5-6}\cmidrule(lr){7-9}
						},
					after row={\midrule},
				},
			% font=\scriptsize,
			columns/n/.style={column type=r,int detect,column name=n},
			columns/m/.style={column type=r,int detect,column name=m},
			columns/first_size_graph_mean_TI/.style={column type=r,fixed,precision=1,zerofill,column name=\emph{Avg Size}},
			columns/first_size_graph_amax_TI/.style={column type=r,fixed,column name=\emph{Max Size}},
			columns/first_size_graph_mean_ATI/.style={column type=r,precision=1,zerofill,fixed,column name=\emph{Avg Size}},
			columns/first_size_graph_amax_ATI/.style={column type=r,fixed,column name=\emph{Max Size}},
			columns/first_size_graph_mean_AFBC/.style={multiply with=2,column type=r,precision=1,zerofill,fixed,column name=\emph{Avg Size}},
			columns/first_size_graph_amax_AFBC/.style={multiply with=2,column type=r,precision=0,fixed,column name=\emph{Max Size}},
			columns/reduction_mean_AFBC/.style={multiply with=100,column type=r,precision=1,zerofill,fixed,column name=\emph{Avg Reduction (\%)}}
		]\summary{}
	\end{threeparttable}
\end{table}


\begin{table}[H]
	\centering
	\begin{threeparttable}
		\caption{Summary of results for algorithms \textbf{TIF}, \textbf{ATIF} and \textbf{BDDF}: computation time LP relaxation}\label{tbl:summarytw}
		\pgfplotstabletypeset[
			columns={n,m,tot_lb_mean_TI,tot_lb_amax_TI,opt_sum_TI,tot_lb_mean_ATI,tot_lb_amax_ATI,opt_sum_ATI,tot_lb_mean_AFBC,tot_lb_amax_AFBC,opt_sum_AFBC},
			every head row/.style={
					before row={%
							\toprule
							\multicolumn{2}{c}{}&  \multicolumn{3}{c}{\textbf{TIF}}& \multicolumn{3}{c}{\textbf{ATIF}} & \multicolumn{3}{c}{\textbf{BDDF}}\\
							\cmidrule(lr){3-5}\cmidrule(lr){6-8}\cmidrule(lr){9-11}
						},
					after row={\midrule},
				},
			columns/n/.style={column type=r,int detect,column name=n},
			columns/m/.style={column type=r,int detect,column name=m},
			columns/tot_lb_mean_TI/.style={column type=r,fixed,precision=2,zerofill,column name=\emph{Avg sec}},
			columns/tot_lb_amax_TI/.style={column type=r,precision=2,zerofill,column name=\emph{Max sec}},
			columns/opt_sum_TI/.style={column type=r,fixed,column name=\emph{\# opt}},
			columns/tot_lb_mean_ATI/.style={column type=r,precision=2,zerofill,fixed,column name=\emph{Avg sec}},
			columns/tot_lb_amax_ATI/.style={column type=r,precision=2,zerofill,fixed,column name=\emph{Max sec}},
			columns/opt_sum_ATI/.style={column type=r,fixed,column name=\emph{\# opt}},
			columns/tot_lb_mean_AFBC/.style={column type=r,precision=2,zerofill,fixed,column name=\emph{Avg sec}},
			columns/tot_lb_amax_AFBC/.style={column type=r,precision=2,zerofill,fixed,column name=\emph{Max sec}},
			columns/opt_sum_AFBC/.style={column type=r,fixed,column name=\emph{\# opt}}
		]\summary{}
	\end{threeparttable}
\end{table}


\begin{table}[H]
	\setlength{\tabcolsep}{4pt}
	\centering
	\begin{threeparttable}
		\caption{Summary of results for algorithms \textbf{TIF}, \textbf{ATIF} and \textbf{BDDF}: gap}\label{tbl:summarytw2}
		\pgfplotstabletypeset[
			columns={n,m,gap_mean_TI,gap_amax_TI,gap_mean_ATI,gap_amax_ATI,gap_mean_AFBC,gap_amax_AFBC},
			every head row/.style={
					before row={%
							\toprule
							\multicolumn{2}{c}{}&  \multicolumn{2}{c}{\textbf{TIF}}& \multicolumn{2}{c}{\textbf{ATIF}} & \multicolumn{2}{c}{\textbf{BDDF}}\\
							\cmidrule(lr){3-4}\cmidrule(lr){5-6}\cmidrule(lr){7-8}
						},
					after row={\midrule},
				},
			% font=\scriptsize,
			columns/n/.style={column type=r,int detect,column name=n},
			columns/m/.style={column type=r,int detect,column name=m},
			columns/gap_mean_TI/.style={multiply with=100,column type=r,precision=2,zerofill,column name=\emph{Avg gap\%}},
			columns/gap_amax_TI/.style={multiply with=100,column type=r,precision=2,zerofill,column name=\emph{Max gap\%}},
			columns/gap_mean_ATI/.style={multiply with=100,column type=r,precision=2,zerofill,fixed,column name=\emph{Avg gap\%}},
			columns/gap_amax_ATI/.style={multiply with=100,column type=r,precision=2,zerofill,fixed,column name=\emph{Max gap\%}},
			columns/gap_mean_AFBC/.style={multiply with=100,column type=r,precision=2,zerofill,fixed,column name=\emph{Avg gap\%}},
			columns/gap_amax_AFBC/.style={multiply with=100,column type=r,precision=2,zerofill,fixed,column name=\emph{Max gap\%}}
		]\summary{}

	\end{threeparttable}
\end{table}

	\foreach \i/\j/\k in {2/40/a,4/40/b,2/50/c,4/50/d,2/100/e,4/100/f,2/150/g,4/150/h}{
			\begin{table}[h]
				\centering
				\caption{Comparison of algorithms \textbf{TIF}, \textbf{ATIF} and \textbf{BDDF} with \(m = \i \) and \(n = \j \)}\label{tbl:tw\k}
				\begin{threeparttable}
					\pgfplotstabletypeset[
						columns={Id,global_upper_bound_TI,global_lower_bound_TI,nb_generated_col_TI,tot_lb_TI,global_lower_bound_ATI,nb_generated_col_ATI,tot_lb_ATI,global_lower_bound_AFBC,nb_generated_col_AFBC,tot_lb_AFBC},
						every head row/.style={
								before row={%
										\toprule
										\multicolumn{2}{c}{}& \multicolumn{3}{c}{\textbf{TIF}} &  \multicolumn{3}{c}{\textbf{ATIF}} & \multicolumn{3}{c}{\textbf{BDDF}}\\
										\cmidrule(lr){3-5}\cmidrule(lr){6-8}\cmidrule(lr){9-11}
									},
								after row={\midrule},
							},
						columns/Id/.style={column type=r,int detect,column name=\#\emph{id}},
						columns/global_upper_bound_TI/.style={column type=r,int detect,column name=\emph{UB}},
						columns/global_lower_bound_TI/.style={column type=r,int detect,column name=\emph{LB}},
						columns/global_lower_bound_ATI/.style={column type=r,int detect,column name=\emph{LB}},
						columns/global_lower_bound_AFBC/.style={column type=r,int detect,column name=\emph{LB}},
						columns/nb_generated_col_TI/.style={column type=r,int detect,column name=\#\emph{iter}},
						columns/nb_generated_col_ATI/.style={column type=r,int detect,column name=\#\emph{iter}},
						columns/nb_generated_col_AFBC/.style={column type=r,int detect,column name=\#\emph{iter}},
						columns/tot_lb_AFBC/.style={column type=r,fixed,precision=1,zerofill,column name=\emph{sec}},
						columns/tot_lb_TI/.style={column type=r,fixed,precision=1,zerofill,column name=\emph{sec}},
						columns/tot_lb_ATI/.style={column type=r,fixed,precision=1,zerofill,column name=\emph{sec}},
						row predicate/.code={
								\pgfplotstablegetelem{##1}{n}\of{\allinstances}
								\pgfmathsetmacro{\xn}{\pgfplotsretval}
								\ifnum\xn=\j
									\pgfplotstablegetelem{##1}{m}\of{\allinstances}
									\pgfmathsetmacro{\yn}{\pgfplotsretval}
									\ifnum\yn=\i
									\else
										\pgfplotstableuserowfalse{}
									\fi
								\else
									\pgfplotstableuserowfalse{}
								\fi
							}
					]\allinstances
				\end{threeparttable}
			\end{table}
		}

\end{document}